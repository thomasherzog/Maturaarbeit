\documentclass[../main.tex]{subfiles}

\begin{document}
	\section{Oktober}
	
	\subsection{Montag, 05.10.2020}
	Während der Zugfahrt teilte ich die Kapitel neu ein, da sich eine bessere Aufteilung ergab.
	Neben dem Kapitel 2 und Kapitel 3 schrieb ich ein neues Kapitel "Analyse".
	
	\subsection{Dienstag, 06.10.2020}
	Ich nutzte meine Zeit um formale Informationen in die Einleitung einzubauen.
	Ich schrieb die Unterkapitel "Aufbau der Arbeit" und "Methodik".
	
	\subsection{Mittwoch, 07.10.2020}
	Ich nutzte den Tag, um ein komplett neues Unterkapitel, das Kapitel \texttt{Rechtslage \& Einschränkungen}, zu beginnen. 
	Der erste Schritt bestand daraus, mich grob ins Alkoholgesetz (AlkG) einzulesen und genau zu verstehen, wie die Besteuerung von Alkohol und Tabakprodukten funktioniert. 
	Nach dem Durchlesen wurde mir bewusst, dass es für die Legalisierung ein neues spezifisches Gesetz für den generellen Umgang bräuchte. Deswegen schrieb ich eine kleine Einleitung in \texttt{Rechtslage \& Einschränkungen > Cannabisgesetz}.
	Die Erkentnisse über das Besteuerungsmodell wende ich dann im Unterkapitel \texttt{Rechtslage \& Einschränkungen > Besteuerung} an.\\
	
	\noindent	
	Um zu verstehen, wie regulierte psychoaktive Stoffe in der Schweiz besteuert werden, habe ich folgende Gesetze gelesen:
	
	\begin{itemize}
		\item \href{https://www.admin.ch/opc/de/classified-compilation/19320035/index.html}{Bundesgesetz über die gebrannten Wasser (Alkoholgesetz, AlkG)}
		\item \href{https://www.admin.ch/opc/de/classified-compilation/19690056/index.html}{Bundesgesetz über die Tabakbesteuerung (Tabaksteuergesetz, TStG)}
	\end{itemize}
	
	
	\subsection{Samstag, 10.10.2020}
	Ich schrieb am Unterkapitel \texttt{Rechtslage \& Einschränkungen} weiter und schrieb zwei neue Unterkapitel.
	Um an den neuen Unterkapitel \texttt{Rechtslage \& Einschränkungen > Jugendschutz} und \texttt{Rechtslage \& Einschränkungen > Werbeeinschränkungen} zu arbeiten, las ich viele Beiträge über die Handhabung von Werbung über Tabak und Alkohol. 
	 
	
	% Weiterschreiben: Jugendschutz, Werbeeinschränkungen
	% Einlesen in Alkoholgesetz und Tabakeinschränkungen
	
	\subsection{Sonntag, 11.10.2020}
	

\end{document}