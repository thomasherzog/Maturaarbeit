\documentclass[../main.tex]{subfiles}

\begin{document}
	\section{Oktober}
	
	\subsection{Mittwoch, 07.10.2020}
	Ich nutzte den Tag, um ein komplett neues Unterkapitel, das Kapitel \texttt{Rechtslage \& Einschränkungen}, zu beginnen. 
	Der erste Schritt bestand daraus, mich grob ins Alkoholgesetz (AlkG) einzulesen und genau zu verstehen, wie die Besteuerung von Alkohol und Tabakprodukten funktioniert. 
	Nach dem Durchlesen wurde mir bewusst, dass es für die Legalisierung ein neues spezifisches Gesetz für den generellen Umgang bräuchte. Deswegen schrieb ich eine kleine Einleitung in \texttt{Rechtslage \& Einschränkungen > Cannabisgesetz}.
	Die Erkentnisse über das Besteuerungsmodell wende ich dann im Unterkapitel \texttt{Rechtslage \& Einschränkungen > Besteuerung} an.\\
	
	\noindent	
	Um zu verstehen, wie regulierte psychoaktive Stoffe in der Schweiz besteuert werden, habe ich folgende Gesetze gelesen:
	
	\begin{itemize}
		\item \href{https://www.admin.ch/opc/de/classified-compilation/19320035/index.html}{Bundesgesetz über die gebrannten Wasser (Alkoholgesetz, AlkG)}
		\item \href{https://www.admin.ch/opc/de/classified-compilation/19690056/index.html}{Bundesgesetz über die Tabakbesteuerung (Tabaksteuergesetz, TStG)}
	\end{itemize}
	
	
	\subsection{Samstag, 10.10.2020}
	Ich schrieb am Unterkapitel \texttt{Rechtslage \& Einschränkungen} weiter und schrieb zwei neue Unterkapitel.
	Um an den neuen Unterkapitel \texttt{Rechtslage \& Einschränkungen > Jugendschutz} und \texttt{Rechtslage \& Einschränkungen > Werbeeinschränkungen} zu arbeiten, las ich viele Beiträge über die Handhabung von Werbung über Tabak und Alkohol. 
	 
	
	% Weiterschreiben: Jugendschutz, Werbeeinschränkungen
	% Einlesen in Alkoholgesetz und Tabakeinschränkungen
	
	\subsection{Sonntag, 11.10.2020}
	
	
	\subsection{Montag, 18.10.2020}
	Ich nutzte den Tag, um am Design der Arbeit zu arbeiten.
	Ich passte sowohl die Seitenränder als auch den Zeilenabstand an. 
	Die neuen Masse sind: 
	\begin{singlespace}
	\noindent
	\begin{itemize}
		\item Seitenabstand: links = 2cm, rechts = 2cm, oben = 3cm, unten = 3cm\\[5pt]
		\verb!\usepackage{geometry}!
		\verb!\geometry{a4paper, left=2cm,right=2cm,top=3cm,bottom=3cm}!
		\item Zeilenabstand: 1.5\\[5pt]
		\verb!\usepackage[onehalfspacing]{setspace}!
	\end{itemize}
	\end{singlespace}
	
	\noindent
	Den Header und Footer habe ich mit dem LaTeX Paket \texttt{fancyhdr} gemacht.
	Die Kopfzeile besteht aus dem Kapitel auf der linken Seite und meinem Namen auf der rechten Seite.
	Die Fusszeile zeigt die Seitenanzahl auf der rechten Seite an.
	Das Layout in der Arbeit ist das gleiche wie im Arbeitsjournal, beide wurden zur gleichen Zeit gemacht.
	
	

\end{document}