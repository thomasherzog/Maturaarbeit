\documentclass[../main.tex]{subfiles}

\begin{document}

	\section{Anhänge}
	
	\subsection{E-Mails}
	
	\subsubsection{E-Mail vom Freitag, 04.09.2020}
	
	\begin{tcolorbox}[title=Thomas Herzog to Frank Zobel]
		Sehr geehrter Herr Zobel\\

		Im Rahmen meiner Maturaarbeit "Cannabis - Die marktwirtschaftliche Legalisierung" bin ich auf Ihre Beiträge zum Cannabismarkt der Schweiz gestossen. Vorab möchte ich Ihnen danken, dass Sie durch Ihre Arbeit meine Matura ermöglichen.\\

		Während dem Durchlesen der Quellen blieben mir jedoch zwei Fragen offen. Die erste Frage bezieht sich auf die Medienmitteilung "Der Cannabis-Markt unter der Lupe", die sich auf Ihre Studie über den Kanton Waadt bezieht. Ich konnte nirgends die Angabe zum ganzen Marktvolumen der Schweiz (40-60 Millionen) in der Studie finden. So denke ich, dass die Berechnungen für die Medienmittelung gemacht wurden. Nach meinen Berechnungen liegt mein Ergebnis jedoch leicht darunter. Ich wäre sehr dankbar, wenn Sie mir Ihr Vorgehen erklären könnten.\\

		Medienmitteilung: \\
		https://www.suchtschweiz.ch/aktuell/medienmitteilungen/article/der-cannabis-markt-unter-der-lupe/\\

		Die zweite Frage bezieht sich auf die Quellenangabe des Durchschnittspreises. Da die persönliche Mitteilung des Bundesamtes für Polizei wahrscheinlich nicht öffentlich zugänglich ist, wollte ich Sie anfragen, ob Sie mir diese zusenden könnten. Die Quellenangabe wäre "Bundesamt für Polizei (fedpol) (2015): persönliche Mitteilung".\\

		Quelle:	https://zahlen-fakten.suchtschweiz.ch/de/cannabis/markt.html\\


		Mit freundlichen Grüssen\\
		Thomas Herzog
	\end{tcolorbox}
	
	\begin{tcolorbox}[title=Frank Zobel zu Thomas Herzog]
		Sehr geehrter Herr Herzog,\\

Hier meine Antworten.\\

Frage 1: Wir haben ganz einfach die im Kanton Waadt geschätzte konsumierte Menge grob anhand Bevölkerungszahlen auf die ganze Schweiz hochgerechnet (für die Medienmitteilung. Es ist also eine sehr grobe und unwissenschaftliche Schätzung. Man sollte sicher auch berücksichtigen, dass im Kt Waadt eher mehr konsumiert wird als im Durchschnitt in der Schweiz. Diese Zahl (Menge) diente hauptsächlich um zu zeigen, dass frühere Schätzungen von etwa 100 Tonnen sehr wahrscheinlich zu hoch waren und das der Cannabismarkt nicht so gross ist wie man manchmal denkt.\\

Frage 2: Diese Daten wurden früher von der Bundespolizei (Fedpol) bei den Kantonspolizei gesammelt. Ich habe keinen direkten Zugang zu den Daten, die ausserdem nicht sehr solide waren. Wir haben innerhalb unseres Websurveys mit Usern bessere Daten über Preise bekommen (sie stehen im Bericht).\\

Ich hoffe es hilft Ihnen und viel Glück mit Ihrer Maturaarbeit.\\

Mit freundlichen Grüssen\\

Frank Zobel\\
Vize Direktor
	\end{tcolorbox}

\end{document}