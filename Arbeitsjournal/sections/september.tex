\documentclass[../main.tex]{subfiles}

\begin{document}
	\section{September}
	
	\subsection{Freitag, 04.09.2020}
	An diesem Tag begann ich mit dem Schreiben der Reinfassung.
	Ich legte alle Dateien in einer sauberen Dateienstruktur auf einem USB Stick ab.
	Sowohl auf meinem Laptop als auch auf meinem Desktop PC habe ich eine zusätzliche lokale Kopie abgespeichert.\\
	\noindent	
	Ich schrieb zudem eine E-Mail an den Vize-Direktor der Organisation Sucht-Schweiz Frank Zobel.
	Der Grund für diese E-Mail war, dass ich einige Informationen in diversen Quellen nicht nachvollziehen konnte.
	Die Antwort von Frank Zobel konnte mehr Klarheit schaffen, so dass ich problemlos weiterschreiben kann.
	Beide E-Mails, die Anfrage und die Antwort, findet man im \hyperlink{Attachment-1.1}{\texttt{Kapitel 4.1.1}}.
	
	
	\subsection{Mittwoch, 09.09.2020}
	Ich habe die Kapitel \texttt{Illegale Drogenmärkte} und \texttt{Legalisierung} angefangen.
	Als Grundlage nutzte ich die Quellen \cite{golzar} und \cite{becker}.
	Ich schrieb die Kapitel \texttt{Illegale Drogenmärkte > Preiselastizität} und \texttt{Legalisierung > Preisniveau} fertig. 
	
	
	\subsection{Mittwoch, 17.09.2020}
	An diesem Tag gab es keine Änderungen am Reintext.
	Es wurde nur administrative und formelle Arbeit geleistet. 
	Ich stieg auf das Typesetting Programm LaTeX um, da ich meine Effizienz stark steigern kann.
	Dank LaTeX wird das Layout standardmässig besser als bei Word angezeigt und kleine Anpassungen kann ich gleich in den Code schreiben.\\
	
	\noindent
	Zuvor habe ich die Arbeit nur lokal auf einem USB Stick und auf mehreren PCs gespeichert.
	Nun habe ich den Code auf GitHub, einem Speicher für Code, hochgeladen.
	Die Arbeit ist nun durch diesen Link öffentlich zugänglich:
	\begin{itemize}
		\item \href{https://github.com/thomasherzog/Maturaarbeit}{https://github.com/thomasherzog/Maturaarbeit}
	\end{itemize}
	
	
	\subsection{Mittwoch, 23.09.2020}
	Ich schrieb die formalen Teile der Einleitung, insgesamt die Teile \texttt{Themenwahl}, \texttt{Aufbau der Arbeit} und \texttt{Methodik}.
	Für diese Kapitel waren keine Quellen nötig.
	
	% Verschiebung von Illegale Drogenmärkte und Cannabiskonsumenten in eigene Sektion
	% Einleitung geschrieben 1 1/2 Seiten	
	
	\subsection{Dienstag, 29.09.2020}
	Ich nutzte den Tag, um Anpassungen an die Struktur vorzunehmen und weitere formale Informationen in die Einleitung einzubauen.
	Ich schrieb den Teil \texttt{Einleitung > Leitfragen}.
	Die neue Struktur sieht wie folgt aus:
	\begin{singlespace}
	\noindent
	\begin{itemize}
		\item \makebox[4.5cm][l]{Vorwort:} 1 Seite
		\item \makebox[4.5cm][l]{Einleitung:} 4 Seiten
		\item \makebox[4.5cm][l]{Illegale Drogenmärkte:} 6 Seiten
		\item \makebox[4.5cm][l]{Legalisierung:} 8 Seiten
		\item \makebox[4.5cm][l]{Analyse:} 3 Seiten
		\item \makebox[4.5cm][l]{Schlussfolgerung:} 1 Seite
	\end{itemize}
	\end{singlespace}
	
\end{document}