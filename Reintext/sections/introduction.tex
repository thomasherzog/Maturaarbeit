\documentclass[../main.tex]{subfiles}

\begin{document}
	 \section{Einleitung}
	 
	 \subsection{Themenwahl}
	 Die Themenwahl verlief für mich sehr einfach, da mir schon von Anfang an bewusst war, dass ich eine wirtschaftliche Maturaarbeit schreiben wollte. 
	 Die Präferenz eine wirtschaftliche Arbeit zu schreiben, mischte ich mit meinen privaten Interessen im Bereich der Drogenpolitik. 
	 Dass die Wahl schlussendlich auf die Droge Cannabis fiel, kann man so erklären, dass der Konsum viele Menschen unserer Gesellschaft betrifft. 
	 Dies konnte ich auch bestätigen, nachdem ich durch die Hilfe von Statistiken die genaueren Zahlen betrachten konnte. 
	 Zur Zeit der Themenfindung waren in Deutschland und teilweise in der Schweiz gesundheitsschädliche synthetische Cannabinoide im Umlauf, was meine Aufmerksamkeit noch weiter auf die Cannabis Legalisierung zog.
	 Anfangs war die Themenwahl sehr breit formuliert und schloss jeden Aspekt einer Legalisierung ein. 
	 Erst beim Erarbeiten der Quellen wurde mir bewusst, wie tiefgründig das Thema ist. 
	 Während der Phase der Erarbeitung der Quellen wurde das Thema immer weiter eingegrenzt, sodass am Ende der Fokus auf dem marktwirtschaftlichen Ansatz der Legalisierung lag. 
	 Das theoretische Konzept einer marktwirtschaftlichen Legalisierung wollte ich dann auf die Schweizer Wirtschaft abbilden.
	 
	 \subsection{Aufbau der Arbeit}
	 Die Einleitung enthält formale Informationen über die Arbeit und bietet dem Leser eine gute Basis für das Verständnis des Inhaltes. 
	 Dem Leser wird ein geschichtlicher Hintergrund über die Prohibition von Cannabis bis hin zur heutigen Zeit vermittelt. 
	 Zudem wird die Schweizer Drogenpolitik in Relation mit anderen Ländern gestellt, um so einen internationalen Vergleich darzustellen.
	 Im nächsten Kapitel wird der zurzeit präsente Drogenmarkt, namentlich der illegale Schwarzmarkt, mit ökonomischen Mitteln analysiert. 
	 In der Analyse soll schon klar werden, aufgrund welchen wirtschaftlichen Faktoren eine Prohibition sich positiv aber auch negativ auf den Konsum der Gesellschaft auswirken kann. 
	 Im dritten Kapitel wird ein Konzept für eine mögliche Legalisierung erarbeitet. 
	 Das Konzept soll dabei möglichst den Werten der Schweizer Rechtsprechung und Politik entsprechen. 
	 %Es werden nötige Einschränkungen + Besteuerung berechnet
	 Im Kapitel "Analyse" werden beide Modelle in Bezug auf die wirtschaftlichen Auswirkungen miteinander verglichen. 
	 % Staatshaushalt und Gesellschaft
	 Die Schlussfolgerung dient dazu, die Ergebnisse abschliessend zusammenzufassen und eine abschliessende Meinung zu bilden.
	 
	 
	 \subsection{Methodik}
	 Die zum grössten Teil vorherrschende Methodik ist die Literaturanalyse, wobei viele Teile auf eigenen Gedankengängen basieren. 
	 So wird für einzelne Zahlen auf bereits vorhandene Studien zurückgegriffen, da spezialisierte Organisationen eine weitaus bessere Datenbasis besitzen und eine höhere Anzahl an Studienteilnehmer befragen können. 
	 Aufgrund des Gesetzes der grossen Zahlen ist es sinnvoll, da so der Wert der Studie viel näher an dem tatsächlichen Wert liegt.
	 Die Basis der Arbeit besteht aus Büchern, theoretisch-wissenschaftlichen Arbeiten rund um das Thema Legalisierung und Studien von unabhängigen Organisationen.
	 
	 \subsection{Leitfragen}
	 
	 \begin{itemize}
	 	\item Wie unterscheiden sich die Legalisierung und die Prohibition wirtschaftlich?
	 	\item Welche Vor- und Nachteile bringt eine marktwirtschaftliche Legalisierung im Gegensatz zu der Prohibition und welchen Herausforderungen muss man sich stellen?
	 	\item Welche Massnahmen muss man treffen, damit eine marktwirtschaftliche Legalisierung den Vorstellungen der Schweizer Politik und Gesellschaft entspricht.
	 \end{itemize}
	
	 
	 \subsection{Geschichte}
	 Die Schweizer Drogenpolitik begann ab den 1920er Jahren an Bedeutung zu gewinnen. 
	 Der Einstieg stellt dabei die Internationale Opiumkonferenz dar. 
	 Die Internationale Opiumkonferenz führte zum ersten internationalen Abkommen über den Umgang mit Betäubungsmitteln. 
	 Die Opiumkonvention führte jedoch nur zu einem Vebot von Morphin und Kokain. 
	 Weitere Verschärfungen der Drogenpolitik kamen erst 1951, als das Bundesgesetz über die Betäubungsmittel und die psychotropen Stoffe (BetmG) verabschiedet wurde.
	 Seit der Einführung des BetmG ist auch Cannabis als verbotenes Betäubungsmittel klassifiziert.\\	 
	 
	 \noindent
	 Die wichtigste Änderung des BetmG wurde im Jahr 2011 durchgeführt, als die Bevölkerung 2008 dem revidiertem Betäubungsmittelgesetz zustimmten. 
	 Seit der Änderung sind die Vier-Säulen-Politik und die Behandlungen mit Heroinabgaben feste Bestandteile. Die Vier-Säulen-Politik sieht vor, dass neben der Repression auch Massnahmen in Prävention, Schadensminderung und Therapie getroffen werden. 
	 So liegt der Schwerpunkt nicht mehr auf der Durchsetzung der Repression, sondern auf dem Wohl der Gesellschaft.
	 Der Konsum wird nicht mehr nur durch das Strafmass gesteuert, sondern auch durch Prävention gemindert. 
	 Bereits erkrankten Menschen werden gesetzlich vorgesehene Massnahmen zur Verfügung gestellt.\\
	 
	 \noindent
	 Zur gleichen Zeit wurde über die Volksabstimmung über die Legalisierung von Cannabis abgestimmt.
	 Die Vorlage wurde jedoch von zwei Dritteln der Bevölkerung abgelehnt.
	 Man befürchtete, dass der Drogentourismus stark zunehmen würde, und wollte nicht internationale Abkommen verletzen.
	 Die Befürworter waren nicht überrascht über den Verlust und nahmen die Diskussion gleich wieder auf. 
	 Die neuen Forderungen bestanden aus den Änderungen, dass der Konsum nur noch mit einer Ordnungsbusse bestraft werden kann. 
	 Dies führte zum ersten expliziten Schritt in Richtung einer Legalisierung von Cannabis, sodass die straffreie geringfügige Menge auf 10 Gramm festgesetzt wurde.
	 
	 
	 \subsection{Heutige Lage}
	 In der jetzigen gesetzlichen Lage ist der Besitz von Cannabis erst ab einer Menge über 10 Gramm strafbar, da es sich dann nicht mehr um eine geringfügige Menge handelt und man animmt, dass es sich nicht mehr um Eigenkonsum handelt.
	 Der Umgang mit geringfügigen Mengen ist in \texttt{Art. 19b BetmG} geregelt.
	 Der Konsum ist nach \texttt{Art. 19a BetmG} jedoch immer noch strafbar und wird mit einer Ordnungsbusse bestraft. 
	 Man kann bei Cannabis inzwischen von einer de facto Legalisierung für Konsumenten reden, da der Konsum gesetzlich gesehen zwar verboten ist, jedoch der Nachweis selten erbracht werden kann. 
	 De jure ist THC-haltiges Cannabis immer noch illegal. 
	 Alle vom BetmG kontrollierten Substanzen befinden sich seit der Revidierung des Betäubungsmittelgesetzes, das von der Bevölkerung im Jahr 2011 angenommen wurde, in der Betäubungsmittelverzeichnisverordnung (BetmVV-EDI). 
	 So unterstehen der Stoff Tetrahydrocannabinol (THC) und teilweise dessen synthetische Analoga der Kontrolle.\\
	 
	 \noindent
	 Cannabis ist mit grossem Abstand die meist konsumierte illegale Droge der Schweiz, obwohl sie dem Betäubungsmittelgesetz untersteht. 
	 Etwa $33.8\%$ der Schweizer Bevölkerung \cite{gmel} konsumierte bereits einmal in ihrem Leben Cannabis. 
	 Bei der jungen Bevölkerung sind die Zahlen noch viel höher, sodass bereits die Mehrheit einmal Cannabis konsumiert hat. 
	 Das Wachstum der Zahlen ist seit Jahren positiv und es macht auch keinen Anschein, dass diese sich in Zukunft ändern werden.
	 Die hohe Zahl an Konsumenten und die Entwicklung lässt die Frage offen, ob die Prohibition ihrem gewüschten Zweck dient oder ihr Ziel verfehlt.
	 
	 % TODO: Ablehnung oder nicht -> (Statistik)
	 
	 \subsection{Weltweiter Vergleich}
	 
	 % TODO: Vergleich mit anderen Länder
	 % Fokus Schweiz
	 % Abbildung von allen Ländern weltweit
	 % https://de.wikipedia.org/wiki/Rechtslage_von_Cannabis
	 
	

\end{document}