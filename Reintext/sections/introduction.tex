\documentclass[../main.tex]{subfiles}

\begin{document}
	 \section{Einleitung}
	 
	 \subsection{Geschichte}
	 Die Schweizer Drogenpolitik begann ab den 1920er Jahren an Bedeutung zu gewinnen. 
	 Der Einstieg stellt dabei die Internationale Opiumkonferenz dar. 
	 Die Internationale Opiumkonferenz führte zum ersten internationalen Abkommen über den Umgang mit Betäubungsmitteln. 
	 Die Opiumkonvention führte jedoch nur zu einem Vebot von Morphin und Kokain. 
	 Weitere Verschärfungen der Drogenpolitik kamen erst 1951, als das Bundesgesetz über die Betäubungsmittel und die psychotropen Stoffe (BetmG) verabschiedet wurde.
	 Seit der Einführung des BetmG ist auch Cannabis als verbotenes Betäubungsmittel klassifiziert.\\	 
	 
	 \noindent
	 Die wichtigste Änderung des BetmG wurde im Jahr 2011 durchgeführt, als die Bevölkerung 2008 dem revidiertem Betäubungsmittelgesetz zustimmten. 
	 Seit der Änderung sind die Vier-Säulen-Politik und die Behandlungen mit Heroinabgaben feste Bestandteile. Die Vier-Säulen-Politik sieht vor, dass neben der Repression auch Massnahmen in Prävention, Schadensminderung und Therapie getroffen werden. 
	 So liegt der Schwerpunkt nicht mehr auf der Durchsetzung der Repression, sondern auf dem Wohl der Gesellschaft.
	 Der Konsum wird nicht mehr nur durch das Strafmass gesteuert, sondern auch durch Prävention gemindert. 
	 Bereits erkrankten Menschen werden gesetzlich vorgesehene Massnahmen zur Verfügung gestellt.\\
	 
	 \noindent
	 Zur gleichen Zeit wurde über die Volksabstimmung über die Legalisierung von Cannabis abgestimmt.
	 Die Vorlage wurde jedoch von zwei Dritteln der Bevölkerung abgelehnt.
	 Man befürchtete, dass der Drogentourismus stark zunehmen würde, und wollte nicht internationale Abkommen verletzen.
	 Die Befürworter waren nicht überrascht über den Verlust und nahmen die Diskussion gleich wieder auf. 
	 Die neuen Forderungen bestanden aus den Änderungen, dass der Konsum nur noch mit einer Ordnungsbusse bestraft werden kann. 
	 Dies führte zum ersten expliziten Schritt in Richtung einer Legalisierung von Cannabis, sodass die straffreie geringfügige Menge auf 10 Gramm festgesetzt wurde.
	 
	 
	 \subsection{Heutige Lage}
	 In der jetzigen gesetzlichen Lage ist der Besitz von Cannabis erst ab einer Menge über 10 Gramm strafbar, da es sich dann nicht mehr um eine geringfügige Menge handelt und man animmt, dass es sich nicht mehr um Eigenkonsum handelt.
	 Der Umgang mit geringfügigen Mengen ist in \texttt{Art. 19b BetmG} geregelt.
	 Der Konsum ist nach \texttt{Art. 19a BetmG} jedoch immer noch strafbar und wird mit einer Ordnungsbusse bestraft. 
	 Man kann bei Cannabis inzwischen von einer de facto Legalisierung für Konsumenten reden, da der Konsum gesetzlich zwar verboten ist, jedoch der Nachweis selten erbracht werden kann. 
	 De jure ist THC-haltiges Cannabis immer noch illegal. 
	 Alle vom BetmG kontrollierten Substanzen befinden sich seit der Revidierung des Betäubungsmittelgesetzes 2011, das von der Bevölkerung angenommen wurde, in der Betäubungsmittelverzeichnisverordnung (BetmVV-EDI). 
	 So unterstehen der Stoff Tetrahydrocannabinol (THC) und teilweise dessen synthetischen Analoga der Kontrolle.\\
	 
	 
	 \noindent
	 Cannabis ist mit grossem Abstand die meist konsumierte illegale Droge der Schweiz, obwohl sie dem Betäubungsmittelgesetz untersteht. 
	 Etwa $33.8\%$ der Schweizer Bevölkerung \cite{gmel} konsumierte bereits einmal in ihrem Leben Cannabis. 
	 Bei der jungen Bevölkerung sind die Zahlen noch viel höher, sodass bereits die Mehrheit einmal Cannabis konsumiert hat. 
	 Das Wachstum der Zahlen ist seit Jahren positiv und es macht auch keinen Anschein, dass diese sich in Zukunft ändern werden.
	 Die hohe Zahl an Konsumenten und die Entwicklung lässt die Frage offen, ob die Prohibition ihrem gewüschten Zweck dient oder ihr Ziel verfehlt.
	 
	 
	 \subsection{Leitfragen}
	 
	 \begin{itemize}
	 	\item Inwiefern ist eine Legalisierung verhältnismässig und gesellschaftlich akzeptiert?
	 	\item Wie unterscheidet sich die marktwirtschaftliche Legalisierung von der Prohibition wirtschaftlich?
	 	\item Welche Vor- und Nachteile bringt eine marktwirtschaftliche Legalisierung und welchen Herausforderungen muss man sich stellen?
	 \end{itemize}

\end{document}