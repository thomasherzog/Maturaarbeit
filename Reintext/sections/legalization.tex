\documentclass[../main.tex]{subfiles}

\begin{document}
	 \section{Die Legalisierung}
	 
	 In diesem Kapitel wird das Modell der marktwirtschaftlichen Legalisierung erläutert. 
	 Die marktwirtschaftliche Legalisierung verfolgt den Ansatz, den Markt so frei wie möglich zu gestalten. 
	 Es werden sowohl Auswirkungen auf die Nachfrage als auch auf den Staatshaushalt analysiert. 
	 
	 \subsection{Preisniveau}
	 
	 Um ein angemessenes Preisniveau zu finden, muss man sich stets den Zielen der Legalisierung bewusst sein. 
	 Durch eine Legalisierung möchte man den Schwarzmarkt zerstören und die Konsumenten weitestgehend in diesen integrieren. 
	 Um dieses Ziel zu erreichen darf man den Preis nicht zu hoch ansetzen, da sich sonst ein neuer Schwarzmarkt bildet. 
	 Man will jedoch nicht die Nachfrage der Gesamtbevölkerung erhöhen. 
	 Dafür darf man den Preis nicht tiefer als das Preisniveau der Prohibition ansetzen. 
	 Das Ziel ist es, die Nachfrage stabil zu halten oder gar zu senken.
	 Durch die Legalisierung kann man ein leicht höheres Preisniveau anstreben und es wird sich kein neuer Schwarzmarkt bilden. 
	 Das liegt daran, dass die Kunden bereit sind, einen höheren Preis zu bezahlen, da im Gegenzug Qualität und Quantität gesichert ist. 
	 Man kann einen Preis von etwa 11.50 CHF anstreben. 
	 Dadurch wird sich die Nachfrage leicht senken und es bleibt ein grösserer Spielraum für die Besteuerung übrig.
	 
	 \subsection{Besteuerung}
	 
	 Einer der Punkte in der Drogendebatte ist, dass durch die Prohibition Kosten für die Rechtsdurchsetzung entstehen, während der Staat durch die Legalisierung Steuern auf Cannabisprodukte erheben könnte. 
	 Die Besteuerung erfolgt auf verschiedenen Stufen und indirekte Steuern werden auch berücksichtigt.
	 
	 \subsubsection{Mehrwertsteuer}
	 Die Mehrwertsteuer ist nach dem Bundesgesetz über die Mehrwertsteuer (MWSTG) geregelt.
	 Nach Art. 25 Abs. 1 MWSTG beträgt der Normalsatz 7.7\%. 
	 Für das ganze Marktvolumen gilt der Normalsteuersatz, da wir von THC-haltigem Cannabis ausgehen, das rein zum Freizeitkonsum dient und keinerlei medizinische Zwecke hat. 
	 Die Mehrwertsteuer ist bereits im Bruttoverkaufspreis von 11.50 CHF enthalten. 
	 So beträgt der Nettoverkaufspreis 10.68 CHF und die Mehrwertsteuer 0.28 CHF pro Gramm. \\
	 
	 \noindent
	 Wenn man mit einem Marktvolumen von 60 Tonnen rechnet, dann beträgt die gesamte Mehrwertsteuer 4’902’000’000 CHF.
	 
	 \subsubsection{Cannabissteuer}
	 Durch eine gezielte Steuer kann man die Nachfrage von Suchtmitteln steuern. Nach Schweizer Rechtslage werden Tabakprodukte und Alkohol nach Menge und Art besteuert. 
	 Das gleiche Konzept lässt sich auf Cannabisprodukte anwenden. 
	 

\end{document}