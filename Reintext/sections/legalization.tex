\documentclass[../main.tex]{subfiles}

\begin{document}
	 \section{Die Legalisierung}
	 
	 In diesem Kapitel wird das Modell der marktwirtschaftlichen Legalisierung erläutert. 
	 Die marktwirtschaftliche Legalisierung verfolgt den Ansatz, den Markt so frei wie möglich zu gestalten. 
	 Es werden sowohl Auswirkungen auf die Nachfrage als auch auf den Staatshaushalt analysiert. 
	 
	 \subsection{Das Konzept}
	 Für eine mögliche Legalisierung existieren viele Modelle. 
	 Das ausgewählte Modell wird als Basis für das Erstellen eines Legalisierungskonzeptes für die Schweiz genutzt.
	 
	 \subsubsection{Internationale Konzepte}
	 % Uruguay -> QuasiMonopol
	 % Spanien -> Cannabis Social Clubs
	 % USA, Kanada -> Marktwirtschaft regelt	 
	 
	 Das Ziel der marktwirtschaftlichen Legalisierung ist es, den Markt so frei wie möglich für die Marktteilnehmer zu gestalten, jedoch mit Einschränkungen gegen Missbrauch entgegenzuwirken.
	 
	 \subsubsection{Auswahl eines Basismodelles}
	 Die Auswahl fällt abschliessend auf den marktwirtschaftlichen Ansatz. 
	 Das Ziel der Legalisierung ist es, den Schwarzmarkt weitesgehend zu verdrängen.
	 Die anderen beiden Modelle geben den Konsumenten gewisse Anreize, dass sie dennoch auf den Schwarzmarkt ausweichen würden. 
	 Die Geschichte hat bereits gezeigt, dass das marktwirtschaftliche Modell bereits funktioniert.
	 So ist die Herstellung, der Besitz und der Konsum von Alkohol und Tabak erlaubt, jedoch nur mit gezielten Einschränkungen.
	 
	 
	 \subsection{Preisniveau}
	 
	 Um ein angemessenes Preisniveau zu finden, muss man sich stets den Zielen der Legalisierung bewusst sein. 
	 Durch eine Legalisierung möchte man den Schwarzmarkt zerstören und die Konsumenten weitestgehend in diesen integrieren. 
	 Um dieses Ziel zu erreichen darf man den Preis nicht zu hoch ansetzen, da sich sonst ein neuer Schwarzmarkt bildet. 
	 Man will jedoch nicht die Nachfrage der Gesamtbevölkerung erhöhen. 
	 Dafür darf man den Preis nicht tiefer als das Preisniveau der Prohibition ansetzen. 
	 Das Ziel ist es, die Nachfrage stabil zu halten oder gar zu senken.
	 Durch die Legalisierung kann man ein leicht höheres Preisniveau anstreben und es wird sich kein neuer Schwarzmarkt bilden. 
	 Das liegt daran, dass die Kunden bereit sind, einen höheren Preis zu bezahlen, da im Gegenzug Qualität und Quantität gesichert ist. 
	 Man kann einen Preis von etwa 11.50 CHF anstreben. 
	 Dadurch wird sich die Nachfrage leicht senken und es bleibt ein grösserer Spielraum für die Besteuerung übrig.
	 
	 \subsection{Rechtslage \& Einschränkungen}
	 
	 \subsubsection{Cannabisgesetz}
	 Die Legalisierung kann nicht ohne Einschränkungen erfolgen und deswegen muss man ein neues Gesetz in Betracht ziehen.
	 Als Beispiel dienen Gesetze über den Umgang mit Tabak und Alkohol.
	 Der gesetzliche Umgang mit Alkohol und mit Tabak, zwei legalen psychoaktiven Substanzen wird in eigenen Gesetzen geregelt. 
	 Aus diesem Grund müsste man für Cannabis ein neues Gesetz mit Verordnungen erlassen, das die Herstellung, Einfuhr und Ausfuhr, Verkauf und Besteuerung regeln würde.	 
	 Im Cannabisgesetz werden alle in den folgenden Untersektionen genannten Einschränkungen weiter konkretisiert.
	 
	 \subsubsection{Besteuerung}
	 Die Cannabissteuer muss schon auf der höchsten Stufe der Normenhierarchie geregelt werden. 
	 Die Rechtsgrundlage der Besteuerung wird der von Alkohol und Tabak gleichen, da es sich bei allen drei Produkten um psychoaktive Substanzen handelt.
	 Die besondere Verbrauchssteuer nach \texttt{Art. 131 Abs. 1 BV} muss so erweitert werden, dass der Bund diese auch auf Cannabis und dessen Produkte erheben kann.
	 Die Einnahmen der Verbrauchssteuern sollen in Prävention und Behandlung von Suchtproblemen aber auch in die vorhandenen Ausgleichskassen fliessen.
	 Mit der Erweiterung von \texttt{Art. 131 Abs. 3 BV} (Prävention und Therapie) und \texttt{Art. 112 Abs. 5 BV} (AHV / IV) mit Cannabis, ist die Grundlage für eine Besteuerung geebnet.\\
	 
	 \noindent
	 Nähere Bestimmungen über die Besteuerung würden im neuen Cannabisgesetz festgelegt werden.
	 Generell existieren drei mögliche Ansätze für die Besteuerung von Cannabis.
	 Man könnte Cannabis und dessen Produkte relativ zum Verkaufspreis, nach Gewicht oder nach THC-Gehalt besteuern.
	 Alle drei Ansätze haben Vor- und Nachteile, jedoch ist eine Mischung aus gewichtsbasierender und THC-gehaltsbasierender Besteuerung zu bevorzugen.
	 Die Mischung von beiden Ansätzen wirkt am natürlichsten, da so künstlich hochgezüchtete Cannabissorten nicht unterstützt werden und die Potenz der Produkte miteinberechnet wird.
	 Eine höhere Besteuerung auf potenteres Cannabis sollte chronische Konsumenten abhalten, immer potenteres Cannabis zu konsumieren, das einen höheren Schaden verursacht. 
	 
	 % https://www.bag.admin.ch/bag/de/home/strategie-und-politik/politische-auftraege-und-aktionsplaene/politische-auftraege-zur-alkoholpraevention/alkoholpolitik/gesetzgebung.html
	 
	 
	 
	 \subsubsection{Jugendschutz}
	 In erster Linie dient der Jugendschutz dem Wohle der Kinder und Jugendliche, dass sie von den Risiken des Betäubungsmittelkonsums geschützt werden und ihre psychische und physische Entwicklung nicht beeinträchtigt wird. 
	 So wird bei Alkohol das Schutzalter von hartem Alkohol bei 18 Jahren festgelegt. 
	 Bei Cannabis würde man auch ein Schutzalter zwischen 18-20 Jahren anstreben. 
	 Dies ist sinnvoll, da die Gehirnentwicklung des Menschen bei ungefähr 20 Jahren weitgehend abgeschlossen ist. 
	 Zwar ist mit 18 Jahren die Hirnentwicklung noch nicht vollständig abgeschlossen, jedoch kann man annehmen, dass erwachsene Menschen ab diesem Alter mehr Eigenverantwortung übernehmen können. 
	 Es wäre kontraproduktiv, einem Teil der grössten Gruppe an Konsumenten den Zugang zum legalen Markt zu verweigern, da diese sonst auf den Schwarzmarkt ausweichen. \\
	 
	 \noindent
	 Der Grundgedanke des Jugendschutzes besteht zwar aus einem Abgabeverbot an Jugendliche, jedoch gelten auch die Massnahmen der Vier-Säulen-Politik auch für den Jugendschutz. 
	 Deswegen folgt als direkte Auswirkung, dass schon konsumierende Jugendliche Hilfe bei der Bewältigung ihrer Sucht zur Verfügung gestellt bekommen. 
	 Neben der Suchthilfe für chronische Konsumenten ist auch die Früherkennung und Intervention von grosser Bedeutung. 
	 Durch eine Legalisierung wird der Konsum nicht mehr stark stigmatisiert, sodass Bezugspersonen die Situationen frühzeitig erkennen und handeln können. 
	 Die Prävention erhalten alle Jugendliche, sodass Kinder und Jugendliche wichtige Kompetenzen erlernen, sich gegen den Konsum von Betäubungsmitteln zu stellen.  
	 
	 \subsubsection{Werbeverbot}
	 % Werbung kann Anreize für den Konsum setzen
	 
	 % Auswirkungen Werbung
	 % https://www.zeit.de/wissen/gesundheit/2020-07/tabakwerbeverbot-alkohol-werbeverbot-konsumverhalten-suchtmittel/komplettansicht
	 
	 % Studien zeigen, dass es auf Konsum auswirkt
	 % (BMJ: MacFayden et al., 2001)
	 % https://www.bmj.com/content/322/7285/513
	 % (Cochrane Systematic Review: Lovato et al., 2011)
	 % https://www.cochranelibrary.com/cdsr/doi/10.1002/14651858.CD003439.pub2/full
	 
	 % Gesetzeslage Alkohol
	 % https://www.bag.admin.ch/bag/de/home/strategie-und-politik/politische-auftraege-und-aktionsplaene/politische-auftraege-zur-alkoholpraevention/alkoholpolitik/gesetzgebung.html
	 
	

	 
	 \subsubsection{Strassenverkehrsgesetz}
	 Es herrscht ein komplettes Verbot von Cannabis, namentlich den Inhaltsstoff THC, da es negative Effekte auf die kognitiven Fähigkeiten des Menschen haben kann.
	 THC wird nach einigen Stunden in THC-Carbonsäure metabolisiert, so lassen sich nach Stunden nur noch dessen Metabolite nachweisen.
	 Diese Metabolite haben jedoch eine viel höhere Halbwertszeit, sodass sie noch Wochen nach dem Konsum nachgewiesen werden können.
	 
	 % https://www.amboss.com/de/wissen/Cannabis_%28Intoxikation_und_Abh%C3%A4ngigkeit%29
	 
	 Da Cannabis, namentlich der Inhaltsstoff THC, negative Effekte auf die kognitiven Fähigkeiten des Menschen haben kann, muss das Strassengesetz (SVG) angepasst werden. 
	 
	 Die Feststellung der Fahrunfähigkeit ist in \texttt{Art. 55 SVG} geregelt.
	 Der Artikel regelt zurzeit jedoch nur die Fahrunfähigkeit aufgrund von Alkoholkonsum. 
	 
	 \subsection{Besteuerung}
	 Einer der Punkte in der Drogendebatte ist, dass durch die Prohibition Kosten für die Rechtsdurchsetzung entstehen, während der Staat durch die Legalisierung Steuern auf Cannabisprodukte erheben könnte. 
	 Die Besteuerung erfolgt auf verschiedenen Stufen und indirekte Steuern werden auch berücksichtigt.
	 
	 \subsubsection{Mehrwertsteuer}
	 Die Mehrwertsteuer ist nach dem Bundesgesetz über die Mehrwertsteuer (MWSTG) geregelt.
	 Nach \texttt{Art. 25 Abs. 1 MWSTG} beträgt der Normalsatz 7.7\%. 
	 Für das ganze Marktvolumen gilt der Normalsteuersatz, da wir von THC-haltigem Cannabis ausgehen, das rein zum Freizeitkonsum dient und keinerlei medizinische Zwecke hat. 
	 Die Mehrwertsteuer ist bereits im Bruttoverkaufspreis von 11.50 CHF enthalten. 
	 So beträgt der Nettoverkaufspreis 10.68 CHF und die Mehrwertsteuer 0.28 CHF pro Gramm. \\
	 
	 \noindent
	 Wenn man mit einem Marktvolumen von 60 Tonnen rechnet, dann beträgt die gesamte Mehrwertsteuer 4’902’000’000 CHF.
	 
	 \subsubsection{Cannabissteuer}
	 Durch eine gezielte Steuer kann man die Nachfrage von Suchtmitteln steuern. Nach Schweizer Rechtslage werden Tabakprodukte und Alkohol nach Menge und Art besteuert. 
	 Das gleiche Konzept lässt sich auf Cannabisprodukte anwenden. 
	 % TODO: weiterschreiben
	 
	 
	 

\end{document}