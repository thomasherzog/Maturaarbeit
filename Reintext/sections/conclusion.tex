\documentclass[../main.tex]{subfiles}

\begin{document}

	\section{Schlussfolgerung}
	Es sprechen viele Punkte für eine Cannabis Legalisierung, aber sie hat nicht nur Vorteile.
	Würde man die Legalisierung von einem ökonomischen Standpunkt betrachten, sieht man nur Vorteile, da man Kosten gegen Einnahmen tauschen kann.
	So würde der Staat Einnahmen durch die Mehrwertsteuer in Höhe von CHF 15'316'000 und die Cannabissteuer in Höhe von CHF 246'697'000 einnehmen.
	Weitere Einnahmen würden durch indirekte Steuern wie die Gewinnsteuer, Einkommensteuer und die Sozialversicherungsbeiträge entstehen.
	Im Gegensatz würden die Staatsausgaben von Polizei und Justiz drastisch sinken, da man nur noch ein Bruchteil von BetmG Verstössen anzeigen müsste.
	Die Händler hätten einen erleichterten Markteinstieg, weswegen sie nicht mehr in den illegalen Markt einsteigen würden.\\
	
	\noindent
	Anders kann man die Legalisierung von einem gesellschaftlichen Standpunkt aus betrachten.
	Bei beiden Ansätzen wird die Nachfrage nach Cannabis mit der Zeit steigen.
	Bei der Legalisierung hingegen greift noch der Konsumenten- und Jugendschutz, weswegen sich die negativen Effekte mehrheitlich auf die erwachsenen Bevölkerung beschränken würde.
	Die Gesundheit des Menschen sollte über jedem wirtschaftlichen Interesse sein.
	Jedoch wird den Menschen, die bereits konsumieren, ein besseres Umfeld für ihren Konsum geboten, da sie so von präventiven und therapeutischen Massnahmen profitieren können.
	Die gesundheitlichen Kosten würden von der Gesellschaft übernommen werden, was vielen Menschen politisch nicht gefällt.
	Diese Kosten würden aber ohnehin nicht so hoch sein, wie die Repression heutzutage verursacht.
	
	
	

\end{document}