\documentclass[../main.tex]{subfiles}

\begin{document}

	\section{Analyse}
		
	\subsection{Staatshaushalt}
		
	\paragraph{Besteuerung}
	Durch die Prohibition kann der Staat keine Einnahmen erzielen.
	Da Cannabis illegal ist und somit alle Prozesse über den Schwarzmarkt laufen, hat der Staat keine Einsicht in den Markt und kann auch keine Gewalt ausüben.
	Anders sieht es im Gegensatz bei der Legalisierung aus.
	Der Staat könnte neu durch den Vertrieb von Cannabis und ähnlichen Produkten passiv Einnahmen erzielen.
	Die Mehrwertsteuer und Gewinnsteuer für Unternehmen würde ohne spezifische gesetzliche Änderungen zugreifen.
	Die Mehrwertsteuer wird nach eigenen Berechnungen auf CHF 15'316'000 geschätzt.
	Dadurch dass nun Unternehmen komplett legal Cannabis verkaufen können, wird der Arbeitsmarkt um viele Plätze erweitert.
	Die Angestellten dieser Unternehmen bezahlen ihre Einkommensteuer und lassen das Geld wieder in die Wirtschaft fliessen.
	Die Einnahmen der Gewinnsteuer, Einkommensteuer und Cannabissteuer lassen sich nicht anhand der vorzufindenden Daten errechnen.
	Der Staat hat die Möglichkeit eine Cannabissteuer zu erheben, bei der das maximale Steueraufkommen CHF 246'697'000 beträgt.
	
	
	
	\paragraph{Staatsausgaben}
	Die Ausgaben der Polizei und Justiz sind nicht einfach berechenbar, da nahezu keine Daten existieren.
	Qualitativ betrachtet kann man erkennen, dass bei einer Legalisierung die Staatsausgaben für Polizei und Justiz sinken würden.
	Die Kosten für die Polizei und Justiz würden sofort sinken, da man nicht mehr alle Cannabishändler nach \texttt{Art. 19 Abs. 1 BetmG} anzeigen müsste.
	Es bleiben nur noch die illegalen Händler übrig, die jedoch durch den erleichterten Markteinstieg hohe Opportunitätskosten hätten.
	Die Anreize, dass ein Marktteilnehmer illegal agieren würde, sind so gering, dass es sich nicht lohnen würde.\\
	
	\noindent
	Rein ökonomisch gesehen bringt eine Legalisierung nur Vorteile, da der Staat mehr finanzielle Mittel erhält, ohne gross Mehrarbeit zu leisten.
	Dies bedeutet nicht, dass der Staat die kompletten Einnahmen als Reingewinn verbuchen könnte.
	Eine Legalisierung würde mehr Arbeit in den drei nicht-repressiven Bereichen der Vier-Säulen-Politik erfordern.
	Die Erhöhung der Ausgaben wäre jedoch in einer viel kleineren Grössenordnung als die Einnahmen der Besteuerung.
	Auch der Anstieg an medizinischen Kosten könnte nicht die Steuereinnahmen der Legalisierung ausgleichen.
	Jedoch kann man die Legalisierung nicht nur ökonomisch betrachten, man muss auch gesellschaftliche Aspekte einbringen.
	
	
	\subsection{Gesellschaft}
	
	\paragraph{Konsum}
	Bei der Prohibition besteht das Problem, dass die Bevölkerung dennoch Cannabis konsumiert, obwohl der Konsum illegal wäre.
	Die Prävalenzzahlen steigen entgegen den Zielen der repressiven Drogenpolitik.
	Bei einer Legalisierung würden die Zahlen auch steigen, da man Cannabis ohne Bedenken erwerben könnte.
	Die Auswirkungen des Wachstums wären die gleichen, jedoch wird der Konsument im legalen Markt tiefergehend geschützt.
	Der Konsumentenschutz könnte im legalen Markt greifen und dafür sorgen, dass die Qualität gesichert ist.
	Kinder und Jugendliche werden vom Jugendschutz geschützt und können so vom Konsum abgehalten werden.
	
	\paragraph{Kriminalität}
	Die Kriminalitätsrate wird positiv durch eine Legalisierung beeinflusst.
	Viele Verzeigungen aufgrund Delikten gegen das Betäubungsmittelgesetz würden wegfallen, da diese Akteure nun legal im Markt agieren können.
	Akteure, die sich nicht an die Einschränkungen und Gesetze halten, werden zwar immer noch angezeigt, jedoch sind diese in der Anzahl weitaus weniger.
	Die Endkonsumenten würden vor der Beschaffungskriminalität geschützt werden und die nötige Therapie erhalten.
	Den Fokus wird auf Prävention und Therapie statt auf Repression gelegt, dass so der Schaden möglichst klein gehalten wird.
	
	\paragraph{Gesundheit}
	Cannabis kann der Gesundheit schaden, bei der Repression als auch bei der Legalisierung.
	Auf den ersten Blick scheint es keine gute Idee zu sein, eine gesundheitsschädliche Substanz zu legalisieren.
	Die Prohibition wäre ein guter Ansatz, würde sich die Gesellschaft an die Einschränkungen halten.
	Da der Staat jedoch nicht uneingeschränkte Macht hat, gibt es immer wieder neue Händler und Konsumenten.
	Die Händler werden durch hohe Margen des Marktes und die Konsumenten durch ihr Verlangen nach einem Rausch angelockt.
	So ist es eine Sache der Unmöglichkeit, die Nachfrage komplett zu unterbinden.\\
	
	\noindent
	Die Legalisierung setzt an einem anderen Punkt an als die Prohibition.
	Da man die Nachfrage kaum stoppen kann, muss man bei den anderen drei Pfeilern der Viersäulenpolitik ansetzen.
	Der Gesellschaft soll ein Verständnis für den risikoarmen Konsum bekommen, sodass die schlimmsten Auswirkungen gemindert werden.
	Ähnliche Massnahmen haben sich bereits beim Alkohol- und Tabakkonsum durchgesetzt und bewiesen.
	Auch in Therapie und Schadensminderung kann man höhere Investitionen tätigen, sodass der Rest der Gesellschaft nicht von der Krankheit anderer Mitmenschen negativ beeinflusst wird.
	
	

	
\end{document}