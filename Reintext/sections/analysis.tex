\documentclass[../main.tex]{subfiles}

\begin{document}
	
	\section{Vergleich}
	
	\subsection{Ökonomie}
	
	\paragraph{Art des Marktes}
	Der zurzeit präsente Cannabismarkt der Schweiz ist ein Schwarzmarkt und ist ein Bestandteil der Schattenwirtschaft.
	Mit einer Legalisierung würde der Schwarzmarkt grösstenteils zerstört werden, so dass Nachfrager keine Anreize mehr hätten, illegal ihr Produkt zu erwerben.
	Das Ende der Illegalität hat zwei Effekte zur Folge. 
	Neu hat der Staat die Möglichkeit, den Markt zu regulieren und eine Besteuerung einzuführen.
	Dies ist nicht im Sinn der marktwirtschaftlichen Legalisierung, jedoch ist die dazu nötige Rechtslage gegeben.
	
	\paragraph{Preis}
	Der durchschnittliche Preis von Cannabis würde nach einer Legalisierung stark sinken.
	Neben den Produktionskosten des Anbaus sind die hohe Marge und der Risikoaufschlag grosse Bestandteile des Preises.
	Der durchschnittliche Preis von Cannabis, der zurzeit bei CHF 11.- liegt, würde auf ein Preisniveau von durchschnittlich CHF 1.94 fallen.
	Innovation wird dank der Legalisierung nicht mehr bei der Anonymität erfolgen, sondern beim Anbau selbst und bei weiteren Produktionsprozessen. 
	Innovativere Produktionsprozesse führen zu einem niedrigeren Aufwand, der sich im Endeffekt auf den Einstandspreis auswirkt.
	
	\paragraph{Angebot}
	Durch die Legalisierung würde die Anzahl der Anbieter stark steigen, was man auch anhand des CBD Marktes der Schweiz erkennen kann.
	Innerhalb eines Jahres stieg die Anzahl der CBD Cannabis Hersteller von 5 auf 490.%
	\footnote{Vgl. \cite{nzz-01}.}	
	Eine ähnliche Entwicklung wird man auch beim Cannabismarkt nach der Legalisierung beobachten können.
	Diese Unternehmen unterscheiden sich zu den illegalen Anbietern, dass sie viel professioneller und kontrollierter geführt werden.
	Sie unterstehen der Rechtslage des Staates und können im Gegensatz zu den Anbietern aus dem Schwarzmarkt für rechtswidriges Verhalten belangt werden, da die Nachfrager nicht mehr illegal handeln.
	Für beide Parteien steht nicht mehr die Vermeidung der Polizei im Vordergrund.
	


	\paragraph{Nachfrage}
	Die Nachfrage nach Cannabis zeigt in den letzten Jahren nur einen leichten Anstieg und würde sich auch weiterhin so entwickeln.
	Nach einer Legalisierung würde die Nachfrage einen starken Anstieg verzeichnen.
	Der Nachfrageschub ist durch die Preiselastizität induziert, da der Preis stark sinken würde.
	Die Nachfrage würde stark im Bereich der 30-Tagesprävalenz steigen, während die Lebenszeitprävalenz nur gemässigt steigen würde.
	Dies liegt daran, dass der Konsum in der Gesellschaft nicht stigmatisiert und schon weit verbreitet ist.
	Der Anstieg der 30-Tagesprävalenz korreliert mit einem Anstieg des Probierkonsums, der sich zu Beginn der Legalisierung einstellen würde.
	
	\subsection{Staat}
	
	\paragraph{Steuereinnahmen}
	Durch eine Legalisierung bekommt der Staat die Möglichkeit seine Steuereinnahmen zu erhöhen, ohne einen wiederkehrenden Mehraufwand zu tätigen.
	Die Mehrwertsteuer greift automatisch ein und würde sich wie in Kapitel 3.4.2 berechnet auf einen Betrag zwischen CHF 5.25 Mio. und CHF 7.685 Mio. belaufen. 
	Für den Staat besteht die Möglichkeit, eine Verbrauchssteuer auf Cannabis zu erheben, auch wenn dies nicht im Sinne der marktwirtschaftlichen Legalisierung liegt.
	Die Sozialkassen profitieren indirekt von der Legalisierung, da automatisch mehr in die Kassen einbezahlt wird, wenn der Arbeitsmarkt vergrössert wird.
	Ohne eine Legalisierung verzichtet der Staat auf alle Steuereinnahmen und die passiven Zuflüsse in die Sozialkassen.
	
	
	\paragraph{Staatsausgaben}
	Der grösste Kostenpunkt der Schweizer Drogenpolitik ist die Viersäulenpolitik.
	Der Staat hat Kosten im Bereich der Repression, aber auch in der Prävention.
	Die Kosten der repressiven Massnahmen bestehen vor allem aus den Aufwänden der Justiz und Polizei.
	Nach einer Legalisierung würden die Staatausgaben der Repression nahezu komplett verschwinden.
	Wenn man eine Legalisierung nach dem Konzept dieser Arbeit durchführen würde, würden lediglich die Kosten für die präventiven Massnahmen steigen.
	Präventive Massnahmen gelten jedoch als viel nachhaltiger, da sie auch den Preis senken, während die Repression den Preis erhöht.
	
	\paragraph{Drogenpolitik}
	Die Schweizer Drogenpolitik besteht aus der Viersäulenpolitik, die eine Kombination aus Repression und Prävention darstellt.
	Das Hauptziel der Politik ist es, die Prävalenzen so stark wie möglich zu senken.
	Die neue Drogenpolitik der Legalisierung sieht es vor, nur Massnahmen im Bereich der Prävention zu tätigen, da diese nicht den Preis in die Höhe treiben.
	Die Marktfreiheit bringt dem Staat eine viel grössere Wissensbasis, was für die nationale Drogenpolitik und für die internationale Zusammenarbeit hilfreich ist.
	
	
	
	\subsection{Gesellschaft}
	
	\paragraph{Sicherheit}
	Im Schwarzmarkt existiert für den Nachfrager keine Sicherheit.
	Sowohl die Qualität als auch die Quantität kann stark variieren, ohne dass der Nachfrager etwas dagegen tun könnte.
	Die Qualität hängt damit zusammen, wie stark der Konsum sich negativ auf die Gesundheit der Konsumenten auswirkt.
	Für Nachfrager besteht die Gefahr, Opfer von Betrug oder Gewaltdelikten zu werden.
	Im legalen Markt ist die Qualität und Quantität gewährleistet, da sich die Anbieter auf dem Markt behaupten müssen.
	Durch die Konkurrenz bekommt der Endkonsument das beste Produkt und schlechtere Anbieter werden aus dem Markt verdrängt.
	
	
	
	\paragraph{Kriminalität}
	Der Schwarzmarkt kriminalisiert automatisch alle Anbieter und einen Teil der Nachfrager des Marktes.
	Die zuvor kriminellen Akteure handeln nicht mehr illegal und haben die Möglichkeit auf legalen Weg ihre Produkte zu erwerben und zu verkaufen.
	Die vom Schwarzmarkt induzierte Kriminalität, vor allem Geldwäscherei, Gewalt und Betrug, würde auf dem Schwarzmarkt verbleiben und könnte nicht auf den legalen Markt übergreifen.
	Die Beschaffungskriminalität würde auf ein Minimum reduziert werden, da der Preis von Cannabis stark sinken würde, so dass der Konsum nicht in einem finanziellen Ruin enden könnte.
	
	
	\subsection{Fazit}
	Die Vielzahl an Konsumenten lässt darauf schliessen, dass die Gesellschaft gegenüber der Legalisierung nicht abgeneigt ist.
	Dies ist eine Voraussetzung, dass man eine Legalisierung überhaupt in Betracht ziehen könnte.
	Der Vergleich beider Ansätze zeigt klar, dass sich eine Legalisierung von einem ökonomischen Standpunkt mehr lohnen würde.
	Sowohl die Privatwirtschaft als auch der Staat kann von der Legalisierung finanziell profitieren.
	Ökonomisch gesehen bringt die Prohibition nur Nachteile, die mit einer Legalisierung grösstenteils bekämpft werden würden.
	Gesellschaftlich führt die Legalisierung zu einer Minderung der Kriminalität und Erhöhung der Sicherheit, was auch der Allgemeinheit zugute kommt, da die Allgemeinheit für die Gesundheitskosten aufkommen muss.\\
	
	\noindent		
	Die Schwäche der marktwirtschaftlichen Legalisierung liegt beim Aspekt Gesundheit.
	Die Wissenschaft ist noch zu keinem abschliessenden Ergebnis gekommen, welche Langzeitfolgen den Konsumenten drohen.
	Weitere Untersuchungen im Bereich der Sucht- und Gesundheitsforschung würden aber den Rahmen der Arbeit sprengen, so dass auf diese verzichtet wird
	Dennoch ist es wichtig, weitere Untersuchungen im Bereich der Gesundheit durchzuführen, bevor man eine Legalisierung politisch durchsetzen würde.
	

	
\end{document}