\documentclass[../main.tex]{subfiles}

\begin{document}
	
	\section{Zusammenfassung}
	
	In der Maturitätsarbeit "Cannabis - die marktwirtschaftliche Legalisierung" geht es darum, den Cannabismarkt der Schweiz genauer zu analyiseren und Annahmen zum Markt nach einer Legalisierung zu tätigen.
	Der erste Teil besteht aus einer mikroökonomischen Analyse über den vorzufindenden Markt.
	Im zweiten Teil wird ein Konzept für eine mögliche Legalisierung erarbeitet und die Besteuerung hypothetisch berechnet.
	Der dritte Teil besteht aus einem Vergleich zwischen dem Schwarzmarkt und dem Markt nach der Legalisierung.\\
	
	\noindent
	Der zurzeit vorzufindende Schweizer Cannabismarkt kann in den Bereich der Schattenwirtschaft, genauer in den Bereich der Schwarzmärkte, eingeteilt werden.
	Ein Anbieter befindet sich generell auf einer höhergestellten Position und hat einen viel grösseren Handlungsspielraum als der Nachfrager.
	Der Schweizer Staat nimmt eine passive Rolle im Markt ein und sorgt mit Repression und präventiven Massnahmen für eine Senkung der Prävalenzen.
	Die vom Bund erschaffene Viersäulenpolitik ist dabei das wichtigste Mittel der Schweizer Drogenpolitik.
	Es wird geschätzt, dass bereits 33.8\% aller Schweizer schon einmal Cannabis probiert haben und sich der Konsum auf die jüngeren Altersklassen konzentriert.
	Bei der jüngeren Bevölkerung (20-34 Jahre) beträgt der Anteil Menschen, die mindestens einmal konsumiert haben, über 50\%.
	Im Gegensatz zu anderen Betäubungsmitteln hat Cannabis eine weitaus höhere Prävalenz.
	Dennoch ist der Konsum, Besitz und Handel von Cannabis illegal.\\
	
	\noindent
	Nach einer Legalisierung würde sich der Schweizer Cannabismarkt stark wandeln.
	Anhand dem Beispiel Kanadas wird verdeutlicht, dass der Verkaufspreis von Cannabis stark sinken würde.
	Zudem wird ermittelt, dass sich der Einstandspreis von einem Gramm Cannabis in einem marktwirtschaftlichen System etwa bei CHF 1.80 befindet.
	Ein Konzept nach den Vorstellungen der Schweizer Politik und Gesellschaft enthält Einschränkungen über den Umgang mit Cannabis.
	In einem eigenen Konzept wird eine Rechtsgrundlage für die Besteuerung, den Jugendschutz, über Werbung und für den Strassenverkehr getätigt.
	Die Berechnung der Besteuerung ergibt, dass ohne Einschränkungen ein Steueraufkommen von CHF 5.25-7.685 Mio. generiert wird.TODO.
	
	
\end{document}