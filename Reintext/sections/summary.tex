\documentclass[../main.tex]{subfiles}

\begin{document}
	
	\section{Zusammenfassung}
	
	In der Maturitätsarbeit "Cannabis - die marktwirtschaftliche Legalisierung" geht es darum, den Cannabismarkt der Schweiz genauer zu analysieren und Annahmen zum Markt nach einer Legalisierung zu tätigen.
	Der erste Teil besteht aus einer mikroökonomischen Analyse über den vorzufindenden Markt.
	Im zweiten Teil wird ein Konzept für eine mögliche Legalisierung erarbeitet und die Besteuerung hypothetisch berechnet.
	Der dritte Teil besteht aus einem Vergleich zwischen dem Schwarzmarkt und dem Markt nach der Legalisierung.\\
	
	\noindent
	Der zurzeit vorzufindende Schweizer Cannabismarkt kann in den Bereich der Schattenwirtschaft, genauer in den Bereich der Schwarzmärkte, eingeteilt werden.
	Ein Anbieter nimmt generell immer eine mächtigere Position ein und hat einen viel grösseren Handlungsspielraum als der Nachfrager.
	Der Schweizer Staat nimmt eine passive Rolle im Markt ein und sorgt mit Repression und präventiven Massnahmen für eine Senkung der Prävalenzen.
	Die vom Bund erschaffene Viersäulenpolitik ist dabei das wichtigste Mittel der Schweizer Drogenpolitik.
	Der Cannabiskonsum ist vor allem in der jüngeren Bevölkerung (20-34 Jahren) präsent.
	Im Gegensatz zu anderen Betäubungsmitteln hat Cannabis eine weitaus höhere Prävalenz.
	Dennoch sind Konsum, Besitz und Handel von Cannabis nach \texttt{Art. 19 BetmG} illegal.\\
	
	\noindent
	Das Ziel der marktwirtschaftlichen Legalisierung ist es, die Allokation der Ressourcen möglichst dem Markt zu überlassen.
	Anhand des Beispiels Kanada wird verdeutlicht, dass der Verkaufspreis von Cannabis stark sinken würde.
	Der ermittelte Einstandspreis eines Gramms Cannabis beläuft sich auf CHF 1.80.
	Das in der Arbeit erstellte Konzept soll so weit wie möglich der Schweizer Politik und Gesellschaft entsprechen.
	So werden Einschränkungen und eine Rechtslage für die Besteuerung, den Jugendschutz, die Werbung und den Strassenverkehr erschaffen.
	Die Einschränkungen bestehen aus dem Einführen einer Verbrauchssteuer auf Bundesebene, einem Jugendschutzalter von 18 Jahren, Werbeeinschränkungen und einem Grenzwert im Strassenverkehr von 8.2 µg/L im Blut.\\
	
	\noindent	
	Der Vergleich zwischen dem Schwarzmarkt und dem legalen Markt hat ergeben, dass sich aus einer ökonomischen Sicht eine Legalisierung mehr lohnen würde.
	Auch wenn durch die Legalisierung die Nachfrage stark steigen würde, haben im Endeffekt alle Akteure des Marktes mehr Vorteile.
	Gesellschaftlich gesehen würde sich dies in einer tieferen Kriminalität und höheren Sicherheit für die Konsumenten widerspiegeln.
	Für den Staat bedeutet dies tiefere Ausgaben und höhere Einnahmen, da die ganze Drogenpolitik über Cannabis erneuert werden würde.
	Der Preis würde auf den Einstandspreis fallen, was einen Nachfrageanstieg auslösen würde.

	
	
\end{document}