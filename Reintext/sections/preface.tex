\documentclass[../main.tex]{subfiles}

\begin{document}

	\addcontentsline{toc}{section}{Vorwort}
	\section*{Vorwort}
	
	\subsection*{Abstract}
	Das Ziel dieser Arbeit ist es, ein Konzept zu einer marktwirtschaftlichen Legalisierung zu erarbeiten und die Grundsätze auf die Schweizer Wirtschaft anzuwenden. 
	Die theoretischen Grundsätze der Ökonomie von Schwarzmärkten werden auf den illegalen Schweizer Cannabismarkt angewendet und analysiert. Anhand den Ökonomischen Grundmitteln wird analysiert, wie sich eine Legalisierung und die Prohibition auf den vorherrschenden illegalen Drogenmarkt und dessen Konsumenten auswirkt.
	Das Konzept der marktwirtschaftlichen Legalisierung wird auf die Schweizer Wirtschaft abgebildet und analysiert.
	
	% ===============================
	% ===============================
	% TODO: Abstract weiterschreiben.
	% ===============================
	% ===============================
	
	
	\subsection*{Themenwahl}
	Die Themenwahl verlief für mich sehr einfach, da mir schon von Anfang an bewusst war, dass ich eine wirtschaftliche Maturaarbeit schreiben wollte. 
	Die Präferenz eine wirtschaftliche Arbeit zu schreiben, mischte ich mit meinen privaten Interessen im Bereich der Drogenpolitik. 
	Dass die Wahl schlussendlich auf die Droge Cannabis fiel, kann man so erklären, dass der Konsum viele Menschen unserer Gesellschaft betrifft. 
	Zur Zeit der Themenfindung waren in Deutschland und teilweise in der Schweiz synthetische gesundheitsschädliche Cannabinoide im Umlauf, was meine Aufmerksamkeit noch weiter auf die Cannabis Legalisierung zog. 
	Anfangs war die Themenwahl sehr breit formuliert und schloss jeden Aspekt einer Legalisierung ein. 
	Erst beim Erarbeiten der Quellen wurde mir bewusst, wie tiefgründig das Thema ist.	
	Während der Phase der Erarbeitung der Quellen wurde das Thema immer weiter eingegrenzt, sodass am Ende der Fokus auf dem marktwirtschaftlichen Ansatz der Legalisierung lag.
	Das theoretische Konzept einer marktwirtschaftlichen Legalisierung wollte ich dann auf die Schweizer Wirtschaft abbilden.
	
	
	\subsection*{Danksagung}
	Ich danke allen Personen, die beim Erstellen dieser Arbeit mitgewirkt haben und ihre Ideen und Verbesserungen einfliessen lassen konnten.
	
	\noindent
	Ganz besonders möchte ich meinem ehemaligen Mathematik Lehrer Andrin Schmidt danken, der mich immer unterstützte und meine Fähigkeiten förderte.
	
	\noindent
	Ich danke meinem Betreuer Reto Ammann, der mir diese Arbeit überhaupt ermöglichen konnte und dessen Tätigkeit als Lehrperson an der Kantonsschule Zürich Nord mir die Basis für diese Arbeit lieferte.
	
	\noindent
	Zudem möchte ich Frank Zobel danken, der mit seiner Arbeit im Bereich der Suchtforschung und Drogenpolitik einen erheblichen Beitrag zur Gesellschaft leistet und wichtige Daten für meine Arbeit zur Verfügung stellt.
	
\end{document}