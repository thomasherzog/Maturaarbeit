\documentclass[../main.tex]{subfiles}

\begin{document}

	% Demographie der Konsumenten
	% Auffassung der Gesellschaft

	\section{Cannabiskonsum in der Schweiz}
	
	\subsection{Demographie der Konsumenten}
	Schon $33.8\%$ der Schweizer Bevölkerung hat bereits einmal in ihrem Leben Cannabis konsumiert. 
	Die Entwicklung zeigt auf, dass diese Zahl auch in Zukunft wachsen wird.
	So stieg die Lebenszeitprävalenz in 5 Jahren (2011 bis 2016) um 5 Prozentpunkte.
	Die Lebenszeitprävalenz stellt dar, wie viele Schweizer mindestens einmal in ihrem Leben Cannabis konsumiert haben. 
	Die 12-Monats-, bzw. 30-Tageprävalenz zeigt, wieviele Prozent der Schweizer Bevölkerung in den letzten 12 Monaten, bzw. 30 Tagen Cannabis konsumiert haben.\\
	% TODO: Grafik mit Lebenszeitprävalenz einfügen	
	
	\noindent
	Die 12-Monatsprävalenz ist von $5.0\%$ auf $7.3\%$ angestiegen, während die 30-Monatsprävalenz keinen signifikanten Anstieg aufzeigt.
	Anhand des Verhältnisses von 12-Monatsprävalenz und 30-Tageprävalenz kann man erkennen, dass der regelmässige chronische Konsum in der Gesellschaft kaum gestiegen ist. 
	Die steigende 12-Monatsprävalenz erklärt jedoch, dass die Bereitschaft zum Probekonsum in der Gesellschaft stark gestiegen ist.
	Man kann annehmen, dass der Probekonsum weiterhin steigen wird und sich dies auch in der Lebenszeitprävalenz zeigen wird.
	% TODO: Grafik mit 12-Monats- und 30-Tagesprävalenz einfügen
	
	
	
	\subsection{Auffassung der Gesellschaft}
	Während der Konsum anderer illegaler Betäubungsmittel stigmatisiert ist, scheint dies bei Cannabis nicht der Fall zu sein.
	Die Mehrheit der jungen Erwachsenen hat bereits mindestens einmal Cannabis probiert.
	
	 
	

\end{document}