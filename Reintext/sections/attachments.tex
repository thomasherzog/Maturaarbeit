\documentclass[../main.tex]{subfiles}

\begin{document}

	\addcontentsline{toc}{section}{Anhänge}
	\section*{Anhänge}
	
	\subsection{Das Konzept}
	 Für eine mögliche Legalisierung existieren viele Ansätze, wie weit eine Legalisierung gehen kann.
	 Eine Legalisierung bedeutet nicht, dass Cannabis ohne Einschränkungen für alle Menschen erlaubt wird.
	 Zurzeit sind es drei Länder, die Cannabis legalisiert haben.
	 Die Länder Südafrika, Uruguay und Kanada haben die Legalisierung mit verschiedenen Ansätzen und Einschränkungen durchgesetzt.
	 Georgien wird in dieser Arbeit nicht zu den Ländern mit einer Legalisierung gezählt, da zwar der Besitz und Konsum legal ist, jedoch der Anbau und Handel illegal bleibt. 
	 In Georgien gibt es für die Konsumenten keinen legalen Weg Cannabis zu besitzen.
	 
	 \paragraph{Südafrika}
	 Im Jahr 2018 entscheid das Verfassungsgericht Südafrikas, dass Cannabis für den Eigengebrauch komplett legalisiert werden sollte.\footcite{zacc} 
	 Durch dieses Gerichtsurteil wurde gezielt nur der private Konsum und der Anbau kleiner Mengen legalisiert. 
	 In der Öffentlichkeit herrscht immer noch ein striktes Konsumverbot. 
	 Der Handel mit Cannabis und Cannabisprodukten bleibt weiterhin verboten und wird mit hohen Strafen bestraft. 
	 Damit werden alle Endkonsumenten entlastet, jedoch kann der südafrikanische Staat wirtschaftlich nur sehr wenig von der Legalisierung profitieren. 
	 Die Durchsetzungskosten der Polizei und Justiz fallen zwar weg, jedoch bleiben mögliche Steuereinnahmen komplett weg.
	 Für Touristen gibt es keinen Weg, Cannabis zu konsumieren, da kein legaler Cannabismarkt unter südafrikanischer Gesetzeslage existieren kann.
	 Das Fehlen des Marktes lässt einen Teil des Schwarzmarktes bestehen.
	 
	 \paragraph{Uruguay}
	 Uruguay ist das erste Land, das Cannabis für die Bevölkerung legalisiert hat.
	 Die Legalisierung von Cannabis in Uruguay erfolgt unter strenger staatlicher Kontrolle.\footcite{fijnaut}
	 Bürger können monatlich maximal 40 Gramm Cannabis in Apotheken beziehen oder ihr eigenes Cannabis in Cannabis Social Clubs anbauen.
	 Der Eigenanbau ist unter der Einschränkung erlaubt, dass man maximal 40 Gramm aus der Ernte gewinnen darf.
	 Konsumenten werden in staatlichen Registern registriert und der Handel wird staatlich streng kontrolliert.
	 Für Ausländer gibt es jedoch keine Möglichkeit Cannabis zu erwerben, da der staatliche Verkauf nur für volljährige Bürger vorgesehen ist.
	 Diese Methode löst einen Teil des Schwarzmarktes auf, jedoch können sich Anreize bilden, dennoch auf den Schwarzmarkt auszuweichen. 
	 Touristen werden dennoch auf den Schwarzmarkt ausweichen, da ihnen keine andere Wahl bleibt.
	 Der Konsum in der Öffentlichkeit ist strikt verboten und steht unter Strafe.
	 	 
	 
	 \paragraph{Kanada}
	 Kanada ist das erste Land, das Cannabis mit einem marktwirtschaftlichen Ansatz legalisierte.
	 Die Einschränkungen können zwar je nach Provinz stark schwanken, jedoch kann man einige Trends erkennen.
	 Die meisten Provinzen erlauben den Cannabiskonsum ab 19 Jahren und erlauben den Konsum überall, wo auch das Rauchen von Tabak erlaubt ist.
	 In den meisten Provinzen existieren keine Limits für die Menge des persönlichen Besitzes.
	 Die Händler unterstehen einer Lizenzierungspflicht und werden nur indirekt eingeschränkt. 
	 Es herrscht ein komplettes Werbeverbot auf Cannabis und die Verpackung muss den Vorlagen entsprechen.
	
\end{document}