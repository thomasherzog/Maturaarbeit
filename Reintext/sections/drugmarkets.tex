\documentclass[../main.tex]{subfiles}

\begin{document}

 	\section{Illegale Drogenmärkte}
	 Die meisten Märkte für psychoaktive Substanzen sind illegal.
	 Diese Märkte sind vor allem für ihre Kriminalität bekannt, namentlich Gewalt und Korruption. 
	 Dieses Kapitel erläutert die theoretische Ökonomie eines allgemeinen Drogenmarktes.
	 Für die Konsumenten gilt ein Motto: der Gewinn der Gegenwart wird zu Lasten der Zukunft maximiert. 
	 Das Gegenteil wäre beim Sport der Fall. Suchtmittel stehen im Konflikt mit dem rational handelndem Menschen, 
	 durch die Sucht handeln sie irrational entgegen den Prinzipien des Homo Oeconomicus.

	\subsection{Preiselastizität}
	Die Preiselastizität zeigt an, wie stark das Angebot oder die Nachfrage auf eine Preisänderung reagiert.
	Bei Cannabis wird die Preiselastizität der Nachfrage wie bei vielen Suchtmitteln sehr unelastisch eingeschätzt $(-1<\varepsilon_{xy}<0)$. 
	Nach Studien befindet sich diese Preiselastizität der Nachfrage zwischen $-0.51$ \cite{golzar} und $-0.418$ \cite{halcoussis}.
	Die Preiselastizität von $-0.51$ bzw. $-0.418$ ist der direkte Auslöser, dass eine Preiserhöhung von $1\%$ einen Nachfragerückgang von $-0.51\%$ bzw. $-0.418\%$ zur Folge hat.\\
	
	\noindent
	Eine Prohibition wirkt aufgrund der unelastischen Nachfragekurve nicht positiv ein.
	Eine Verschärfung der Strafverfolgung der Händler und Produzenten bewirkt keinen grossen Rückgang der Nachfrage, jedoch eine Erhöhung des Preises.
	Die Händler schlagen auf ihre Preise einen Risikozuschlag auf und wälzen diesen an ihre Kunden ab.
	Da die Konsumenten aufgrund ihrer Sucht jedoch nicht auf ihr Gut verzichten können, sinkt die Nachfrage kaum. \\
	
	\noindent
	Aus ökonomischer Sicht bringt die Prohibition nur bedingt einen Erfolg in der Bekämpfung des Drogenkonsums. 
	Auf den ersten Blick scheint sie die gewünschte Wirkung zu zeigen, jedoch entsteht dadurch ein anderer Nebeneffekt. 
	Dadurch dass die Preise steigen und die Nachfrage kaum zurückgehen kann, sind die Konsumenten gezwungen die hohen Geldsummen zu bezahlen. 
	Dies führt zu einer erhöhten Beschaffungskriminalität der unteren Gesellschaftsschicht, somit ergibt sich eine erhöhte Kriminalität. 
	Eine erhöhte Kriminalität liegt nicht im Interesse der Allgemeinheit und steht somit dem Grundsatz des öffentlichen Interesses entgegen.
	
	\subsection{Der Schweizer Cannabismarkt}
	Die Nachfrage der Schweizer Cannabiskonsumenten wird hauptsächlich vom Schwarzmarkt bedient. 
	Dies führt dazu, dass keine genauen Daten zum Markt existieren und man auf Schätzungen und Studien zurückgreifen muss.
	
	\subsubsection{Marktvolumen}
	
\end{document}